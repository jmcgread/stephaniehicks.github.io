\documentclass[10pt]{article}

\usepackage{calc}
\usepackage{paralist} % This gives us fun enumeration environments. compactitem will be nice.
\usepackage{enumitem}

\reversemarginpar % Layout: Puts the section titles on left side of page

%% Use these lines for letter-sized paper
\usepackage[paper=letterpaper,
            %includefoot, % Uncomment to put page number above margin
            marginparwidth=1.0in,     % Length of section titles
            marginparsep=.05in,       % Space between titles and text
            margin=.70in,               % 1 inch margins
            includemp]{geometry}

\setlength{\parindent}{0in} %% More layout: Get rid of indenting throughout entire document


\usepackage{fancyhdr,lastpage}
\pagestyle{fancy}
%\pagestyle{empty}      % Uncomment this to get rid of page numbers
\fancyhf{}\renewcommand{\headrulewidth}{0pt}
\fancyfootoffset{\marginparsep+\marginparwidth}
\newlength{\footpageshift}
\setlength{\footpageshift}
          {0.5\textwidth+0.5\marginparsep+0.5\marginparwidth-2in}
\lfoot{\hspace{\footpageshift}%
       \parbox{4in}{\, \hfill %
%                    \arabic{page}  of \protect\pageref*{LastPage} % +LP
                    \arabic{page}                               % -LP
                    \hfill \,}}

% Finally, give us PDF bookmarks
\usepackage{color,hyperref}
\definecolor{darkblue}{rgb}{0.0,0.0,0.3}
\hypersetup{colorlinks,breaklinks,
            linkcolor=darkblue,urlcolor=darkblue,
            anchorcolor=darkblue,citecolor=darkblue}


%%%%%%%%%%%%%%%%%%%%%%%%%%% Helper Commands %%%%%%%%%%%%%%%%%%%%%%%%%%%%

% The title (name) with a horizontal rule under it. e.g. Usage: \makeheading{name}
\newcommand{\makeheading}[1]%
        {\hspace*{-\marginparsep minus \marginparwidth}%
         \begin{minipage}[t]{\textwidth+\marginparwidth+\marginparsep}%
                {\large \bfseries #1}\\[-0.15\baselineskip]%
                 \rule{\columnwidth}{1pt}%
         \end{minipage}}

% The section headings. e.g. Usage: \section{section name}
\renewcommand{\section}[2]%
        {\pagebreak[2]\vspace{1.3\baselineskip}%
         \phantomsection\addcontentsline{toc}{section}{#1}%
         \hspace{0in}%
         \marginpar{
         \raggedright \scshape #1}#2}

% An itemize-style list with lots of space between items
\newenvironment{outerlist}[1][\enskip\textbullet]%
        {\begin{itemize}[#1]}{\end{itemize}%
         \vspace{-.6\baselineskip}}

% An environment IDENTICAL to outerlist that has better pre-list spacing
% when used as the first thing in a \section
\newenvironment{lonelist}[1][\enskip\textbullet]%
        {\vspace{-\baselineskip}\begin{list}{#1}{%
        \setlength{\partopsep}{0pt}%
        \setlength{\topsep}{0pt}}}
        {\end{list}\vspace{-.6\baselineskip}}

% An itemize-style list with little space between items
\newenvironment{innerlist}[1][\enskip\textbullet]%
        {\begin{compactitem}[#1]}{\end{compactitem}}

% An environment IDENTICAL to innerlist that has better pre-list spacing
% when used as the first thing in a \section
\newenvironment{loneinnerlist}[1][\enskip\textbullet]%
        {\vspace{-\baselineskip}\begin{compactitem}[#1]}
        {\end{compactitem}\vspace{-.6\baselineskip}}

% To add some paragraph space between lines.
% This also tells LaTeX to preferably break a page on one of these gaps
% if there is a needed pagebreak nearby.
\newcommand{\blankline}{\quad\pagebreak[2]}

% Uses hyperref to link DOI
\newcommand\doilink[1]{\href{http://dx.doi.org/#1}{#1}}
\newcommand\doi[1]{doi:\doilink{#1}}


%%%%%%%%%%%%%%%%%%%%%%%% End Helper Commands %%%%%%%%%%%%%%%%%%%%%%%%%%%


%%%%%%%%%%%%%%%%%%%%%%%%% Begin CV Document %%%%%%%%%%%%%%%%%%%%%%%%%%%%

\begin{document}
\makeheading{Stephanie C. Hicks \hfill Updated \today}

\section{Contact Information}
%
% NOTE: Mind where the & separators and \\ breaks are in the following
%       table.
%
% ALSO: \rcollength is the width of the right column of the table
%       (adjust it to your liking; default is 1.85in).
%
\newlength{\rcollength}\setlength{\rcollength}{2.05in}%
%
\begin{tabular}[t]{@{}p{\textwidth-\rcollength}p{\rcollength}}
\href{http://bcb.dfci.harvard.edu}{Department of Biostatistics and Computational Biology} & \\
\href{http://www.dana-farber.org}{Dana-Farber Cancer Institute} & twitter: @stephaniehicks \\ 
450 Brookline Avenue   & email: \href{mailto:shicks@jimmy.harvard.edu}{shicks@jimmy.harvard.edu} \\ %Fax: 713.348.5476  \\
Boston, MA 02215 USA   & website: \href{www.stephaniehicks.com}{stephaniehicks.com} 
\end{tabular}

%\href{http://www.hsph.harvard.edu}{Harvard School of Public Health} 

%\section{Objective}
%%
%Placement in a post-doctoral or academic faculty research position
%Research and develop statistical and population genetics tools in application for high-throughput sequencing technologies%to better understand genetic variation within and among populations and its association with phenotypic variation
% Placement in a post-doctoral or academic faculty research position researching and developing statistical and population genetics tools in application for high-throughput sequencing technologies

%\section{Objective} 
%\begin{bibsection}
%\item Placement in an interdisciplinary post-doctoral position researching and developing statistical and bioinformatics methodologies and tools for the analysis of high-throughput genomic data
%\end{bibsection}


%%%%%%%%%%%%%%%%%%%%%%%%%%%%%%%%%%%%%%%%%%
%%%%%%%%%% Research Interests %%%%%%%%%%%%%%%%%%
%%%%%%%%%%%%%%%%%%%%%%%%%%%%%%%%%%%%%%%%%%

\section{Research Interests}
My research interests focus around developing statistical methods and tools in application for genomics and epigenomics data. Currently, I am focused on methods for processing and analyzing DNA methylation and gene expression using microarrays and next-generation sequencing. 

%\section{Teaching Interests}
%In addition to teaching traditional probability and statistics courses, I am particularly interested in developing more diverse courses in a Statistics or Biostatistics department with the goal of teaching statistical methods and tools for reproducible and replicable research to meet the growing demand of data scientists by offering non-theoretical courses targeted at individuals wanting to learn how to become a data scientist. 


%%%%%%%%%%%%%%%%%%%%%%%%%%%%%%%%%%%%%%%%%%%%%%%%%%
%%%%%%%%%%%%%% Current Appointment %%%%%%%%%%%%%%%%%%
%%%%%%%%%%%%%%%%%%%%%%%%%%%%%%%%%%%%%%%%%%%%%%%%%%


\section{Current Appointment}
\textbf{Postdoctoral Research Fellow}, Boston MA  \hfill 2013 - Present
\begin{innerlist}
\item[] Department of Biostatistics and Computational Biology, Dana-Farber Cancer Institute
\item[] Department of Biostatistics, Harvard School of Public Health
\item[] \textit{Mentor: Rafael Irizarry} 
\end{innerlist}

%%%%%%%%%%%%%%%%%%%%%%%%%%%%%%%%%%%%%%%
%%%%%%%%%%%%%% Eduction %%%%%%%%%%%%%%%%%%
%%%%%%%%%%%%%%%%%%%%%%%%%%%%%%%%%%%%%%%

\section{Education}
\vskip -.25in
\begin{table}[h]
\begin{tabular}{cl}
2007 - 2013 & \textbf{PhD. $\&$ M.A., Statistics}, Rice University, Houston, TX USA \\
& \textit{Thesis Advisors: Marek Kimmel, Ph.D. (Rice, Statistics) and } \\
& \hspace{0.95in} \textit{  Sharon Plon, M.D., Ph.D. (Baylor College of Medicine) } \\
& \textit{Thesis Title: Probabilistic Models for Genetic and Genomic Data with Missing Information} \\
& \\
2003 - 2007 & \textbf{B.S., Mathematics}, Louisiana State University, Baton Rouge, LA USA \\
&  \textit{magna cum laude}, Phi Beta Kappa 
\end{tabular}
\end{table}


%%%%%%%%%%%%%%%%%%%%%%%%%%%%%%%%%%%%%%%
%%%%%%%%%%%%%% Professional Experience %%%%%%%%%%%%
%%%%%%%%%%%%%%%%%%%%%%%%%%%%%%%%%%%%%%%
\section{Professional Experience}
\vskip -.25in
\begin{table}[h]
\begin{tabular}{ll}
2013 - & \textbf{Postdoctoral Research Fellow}, DFCI and HSPH, Boston, MA   \\
& \textit{Mentor: Rafael Irizarry} \\
& \\
2007 - 2013 &  \textbf{Graduate Student Researcher}, Rice University, Houston, TX \\
& \textit{Advisors: Marek Kimmel and Sharon Plon (Baylor College of Medicine)} \\
& \\
2006 &  \textbf{Undergraduate Researcher (REU)}, Statistics, University of Wisconsin, Madison, WI \\
& \textit{Mentors: Murray  Clayton and Jo Handelsman} \\
& \\
2005 &  \textbf{Undergraduate Researcher (REU)}, Statistics,  Rice University, Houston, TX  \\
& \textit{Mentor: Javier Rojo} \\
& \\
2005 - 2007 &  \textbf{Undergraduate Researcher}, Psychology, LSU, Baton Rouge, LA  \\
& \textit{Mentor: Robert Mathews} \\
& \\
2005 - 2007 &  \textbf{Undergraduate Tutor}, Roadmap 2 Redesign Program, LSU, Baton Rouge, LA \\
& \textit{ Tutored college algebra, trigonometry and pre-calculus students.} \\
& \\
2003 - 2004 &  \textbf{Undergraduate Tutor}, LSU, Baton Rouge, LA \\
& \textit{ Tutored students at Scotlandville Magnet High School in algebra for 3hrs/week. } \\
\end{tabular}
\end{table}


%%%%%%%%%%%%%%%%%%%%%%%%%%%%%%%%%%%%
%%%%%%%%%%% Publications %%%%%%%%%%%%%%%%%%
%%%%%%%%%%%%%%%%%%%%%%%%%%%%%%%%%%%%


\section{Research}  
\textbf{Journal Articles}
\begin{enumerate}[label= {[\arabic*]}]

\item Liang MK, Li LT, Nguyen MT, Berger RL, {\bf Hicks SC}, Kao LS. (2014). Abdominal reoperation and mesh explantation following open ventral hernia repair with mesh. {\it Am J Surg} [Epub ahead of print] PMID: 25241955. 
\item Liang MK, Berger RL, Nguyen MT, {\bf Hicks SC}, Li LT, Leong M. (2014). Outcomes with Porcine Acellular Dermal Matrix versus Synthetic Mesh and Suture in Complicated Open Ventral Hernia Repair. {\it Surg Infect} [Epub ahead of print] PMID: 25215466. 
\item Berger RL, Li LT, {\bf Hicks SC}, Liang MK. (2014). Suture versus preperitoneal polypropylene mesh for elective umbilical hernia repairs. {\it J Surg Res} [Epub ahead of print] PMID: 24980854 
\item Brahmbhatt R, Carter SA, {\bf Hicks SC}, Berger DH, Liang MK. (2014). Identifying Risk Factors for Surgical Site Complications after Laparoscopic Ventral Hernia Repair: Evaluation of the Ventral Hernia Working Group Grading System. {\it Surg Infect (Larchmt)} {\bf 15}: 187-193. PMID: 24773169
\item Carter SA, {\bf Hicks SC}, Brahmbhatt R, Liang MK. (2014). Recurrence and Pseudorecurrence after Laparoscopic Ventral Hernia Repair: Predictors and Patient-focused Outcomes. {\it Am Surg} {\bf 80}: 138-148. PMID: 24480213
\item Li LT, {\bf Hicks SC}, Davila JA, Kao LS, Berger RL, Arita NA, Liang MK. (2014). Circular Closure is Associated with the Lowest Rate of Surgical Site Infection Following Stoma Reversal:A Systematic Review and Multiple Treatment Meta-analysis. {\it Colorectal Dis} {\bf 16}: 406-416. PMID: 24422861
\item Li LT, Brahmbhatt R, {\bf Hicks SC}, Davila JA, Berger DH, Liang MK. (2014). Prevalence of Surgical Site Infection at the Stoma Site following Four Skin Closure Techniques: A Retrospective Cohort Study. {\it Dig Surg} {\bf 31}: 73-78. PMID: 24776653
\item Nguyen MT, Berger RL, {\bf Hicks SC}, Davila JA, Li LT, Kao LS, Liang MK. (2014). Comparison of Outcomes of Synthetic Mesh vs Suture Repair of Elective Primary Ventral Herniorrhaphy: A Systematic Review and Meta-analysis. {\it JAMA Surg} {\bf 31}: 73-78. PMID: 24554114 
\item Nguyen MT, Phatak UR, Li LT, {\bf Hicks SC}, Moffett JM, Arita NA, Berger RL, Kao LS, Liang MK. (2014). Review of stoma site and midline incisional hernias after stoma reversal. {\it J Surg Res} {\bf 190}: 504-509. PMID: 24560428 
\item Berger RL, Li LT, {\bf Hicks SC}, Davila JA, Kao LS, Liang MK. (2013). Development and validation of a risk-stratification score for surgical site occurrence and surgical site infection after open ventral hernia repair. {\it J Am Coll Surg.} {\bf 217}: 974-982. PMID: 24051068
\item Clapp ML, \textbf{Hicks SC}, Awad SS, Liang MK. (2013). Trans-cutaneous Closure of Central Defects (TCCD) in Laparoscopic Ventral Hernia Repairs (LVHR). {\it World J Surg}. {\bf 37}: 42-51. PMID: 23052806.
\item Li LT, Jafrani RJ, Becker NS, Berger RL,  {\bf Hicks SC}, Davila JA, Liang MK. (2013). Outcomes of acute versus elective primary ventral hernia repair. {\it J Trauma Acute Care Surg} {\bf 76}: 523-528. PMID: 24458061
\item Liang MK, Clapp M, Li LT, Berger RL, {\bf Hicks SC}, Awad S. (2013). Patient Satisfaction, Chronic Pain, and Functional Status following Laparoscopic Ventral Hernia Repair. {\it World J Surg}. {\bf 37}: 530-537. PMID: 23212794. 
\item Liang MK, Li LT, Avellaneda A, Moffett JM, {\bf Hicks SC}, Awad SS. (2013). Outcomes and Predictors of Incisional Surgical Site Infection in Stoma Reversal. {\it JAMA Surg} {\bf 148}: 183-189. PMID: 23426597. 
\item Liang MK, Berger RL, Li LT, Davila JA, {\bf Hicks SC}, Kao LS. (2013). Outcomes of laparoscopic vs open repair of primary ventral hernias. {\it JAMA Surg} {\bf 148}: 1043-1048. PMID: 24005537 
\item Nguyen MT, Li LT, {\bf Hicks SC}, Davila JA, Suliburk JW, Leong M, Kao LS, Berger DH, Liang MK. (2013). Readmission following open ventral hernia repair: incidence, indications, and predictors. {\it Am J Surg} {\bf 206}: 942-948. PMID: 24296099
\item Subramanian A, Clapp ML, {\bf Hicks SC}, Awad SS, Liang MK. (2013). Laparoscopic ventral hernia repair: Primary versus secondary hernias. {\it J Surg Res} {\bf 181}: e1-5. PMID: 22795342.
\item Cheung HC\footnote{ `Highly Accessed' on BioMed Central}, San Lucas FA, \textbf{Hicks S}, Chang K, Bertuch AA, Ribes-Zamora A. (2012). An S/T-Q cluster domain census unveils new putative targets under Tel1/Mec1 control. {\it BMC Genomics} {\bf 13}: 664. PMID: 23176708. 
\item \textbf{Hicks S}, Plon SE, Kimmel M. (2012). Statistical Analysis of Missense Mutation Classifiers. {\it Hum Mut}. e-pub 10/2012. PMID: 23086893. 
\item \textbf{Hicks S}\footnote{Cited and Discussed in Nature {\bf 482}: 257-262. 09 Feb 2012. PMID: 22318607}, Wheeler DE, Plon SE, Kimmel M. (2011). Prediction of Missense Mutation Functionality Depends on both the Algorithm and Sequence Alignment Employed. {\it Hum Mut}. {\bf 32}: 661-668. PMID: 21480434.
\end{enumerate}



\vskip .2in
\textbf{Invited Talks} 
\begin{enumerate}[label= {[\arabic*]}]
	\item Normalization of DNA methylation and Gene Expression Data in the Context of Global Variation.  Bioinformatics Meeting, Division of Immunology, Harvard Medical School. 2014 Sept 18. Boston, MA. 
\end{enumerate}


\vskip .2in
\textbf{Conference Oral Presentations} 
\begin{enumerate}[label= {[\arabic*]}]
	\item {\bf Hicks SC}, Irizarry R. quantro: When should you use quantile normalization? Flashlight talk at \emph{Bioconductor Conference}. 2014 July 30-August 1. Boston, MA. 
	\item {\bf Hicks S}, Plon SE, Kimmel M. Modeling Discovery Of Functional SNPs From Genome Scale Data. Oral presentation: \emph{Joint Statistical Meetings}. 2011 Aug 5. Miami, FL. 
	\item {\bf Hicks S}, Plon SE, Wheeler DE, Kimmel M. Prediction of Missense Mutation Functionality Depends on both the Algorithm and Sequence Alignment Employed. Oral presentation: \emph{Human Genome Variation Society's Exploring the Functional Consequences of Genomic Variation Meeting}. 2010 Nov 1. Washington, D.C.
\end{enumerate}



\vskip .2in

\textbf{Selected Poster Presentations} 
\begin{enumerate}[label= {[\arabic*]}]
\item {\bf Hicks S}, Irizarry R. When to use Quantile Normalization? Poster presented at: \emph{2014 Women in Statistics Conference}. 2014 May 15-17. Cary, NC.
\item {\bf Hicks S}, Irizarry R. Estimating cell composition of whole blood using differentially methylated regions. Poster presented at: \emph{2013 PQG Conference}. 2013. Nov 14-15. Boston, MA. 
\item {\bf Hicks S}, Plon SE, Kimmel M. postMUT: A Statistical Tool for Combining Predictions of Missense Mutation Functionality using Capture-Recapture Methods. Poster presented at: \emph{63nd Annual Meeting of The American Society of Human Genetics}. 2013. Oct 22-26. Boston, MA.
\item Kimmel M, {\bf Hicks S}, Plon SE. Applications of Hidden Markov Models with Conditional Emission Probabilities to Identify Regions of Identity-By-Descent in Whole-Exome Sequencing Data. Poster presented at: \emph{63nd Annual Meeting of The American Society of Human Genetics}. 2013. Oct 22-26. Boston, MA.
\item {\bf Hicks S}, Plon SE, Kimmel M. Capture-Recapture Models for Evaluation of Algorithms Estimating Functionality of Missense Mutations. Poster presented at: \emph{62nd Annual Meeting of The American Society of Human Genetics}. 2012 Nov 6-12. San Francisco, CA. %(Abstract $\#$ 3549F)
\item {\bf Hicks S}, Plon SE, Kimmel M. Bernoulli mixture models in application of the evaluation of algorithms estimating functionality of missense mutations. Poster presented at: \emph{Beyond The Genome}. 2012 Sept 27-29. Boston, MA.
\item {\bf Hicks S}, Plon SE, Kimmel M. Bernoulli Mixture Models in Application of the Evaulation of Algorithms Estimating Functionality of Missense Mutations. Poster presented at: \emph{International Conference in Stochastic Processes}. 2012. Aug 21-25. Houston, TX. 
\item {\bf Hicks S}, Plon SE, Kimmel M. Using a Second-order Hidden Markov Model to Identify Regions of Identity-By-Descent in Exome Sequencing Data. Poster presented at: \emph{12th International Congress of Human Genetics/61st Annual Meeting of The American Society of Human Genetics}. 2011 Nov 12-15. Montreal, Canada.
\item {\bf Hicks S}, Plon SE, Wheeler DE, Kimmel, M. Prediction of Missense Mutation Functionality Depends on both the Algorithm and Sequence Alignment Employed. Poster presented at: \emph{60th Annual Meeting of the American Society of Human Genetics}. 2010 Nov 2-6. Washington, D.C.
\item {\bf Hicks S}, Plon SE, Wheeler DE, Kimmel, M. Prediction of Missense Mutation Functionality Depends on both the Algorithm and Sequence Alignment Employed. Poster presented at: \emph{4th Annual Meeting of the Genomics of Common Diseases}. 2010 Oct 6-9. Houston, TX.
\item Cheung HC, San Lucas F, {\bf Hicks S}, Chang K, Plon SE, Bertuch AA, Ribes-Zamora A. S/T-Q Cluster Domains in Saccharomyces cerevisiae: a bioinformatic census and analysis. Poster presented at: \emph{BioScience Research Collaborative Grand Opening}. 2010 Apr 1. Houston, TX. 
\item {\bf Hicks S}, Plon SE, Wheeler DE, Kimmel, M. Predicting Functionality of Missense Mutations with Varying Levels of Evolutionary Depth. Poster presented at: \emph{Computational \& Theoretical Biology Symposium}. 2009 Dec 4-6. Houston, TX.  
\item {\bf Hicks S}, Plon SE, Wheeler DE, Kimmel, M. Predicting Functionality of Missense Mutations with Varying Levels of Evolutionary Depth. Poster presented at: \emph{59th Annual Meeting of the American Society of Human Genetics}. 2009 Oct 20-24. Honolulu, HI.
\end{enumerate}

\vskip .2in

\textbf{Ph.D. Dissertation}
\begin{enumerate}[label= {[\arabic*]}]
\item {\bf Hicks SC}. Probabilistic Models for Genetic and Genomic Data with Missing Information, Ph.D. Thesis, Rice University, 2013. 
\end{enumerate}

%\vskip .2in


%%%%%%%%%%%%%%%%%%%%%%%%%%%%%%%%%%%%%%
%%%%%%%%%%%%%% Awards %%%%%%%%%%%%%%%%%%
%%%%%%%%%%%%%%%%%%%%%%%%%%%%%%%%%%%%%%


\section{Grants and Awards}
\vskip -0.25in
\begin{table}[h]
\begin{tabular}{ll}
2014 & Travel award for the Women in Statistics Conference 2014 \\
2011 & Travel and tuition award for the 16th Annual Summer  Institute in Statistical Genetics \\ 
& at University of Washington (Population Genetics, Coalescent Theory modules) \\
2008 - 2011 & NIH T32 Training Grant Fellow, Rice University \\
2007 & Inducted into Phi Beta Kappa \\
2007 & LSU Austin Chapter Scholarship Award \\
2005 - 2007 & LA-STEM Research Scholars, LSU \\
2004 - 2005 & HMM Professors Program, LSU \\
2003 - 2007 & TOPS Tuition Award, LSU \\
\end{tabular}
\end{table}

%%%%%%%%%%%%%%%%%%%%%%%%%%%%%%%%%%%%
%%%%%%%%% Teaching Experience %%%%%%%%%%%%%%%%
%%%%%%%%%%%%%%%%%%%%%%%%%%%%%%%%%%%%


\section{Teaching Experience}
\vskip -.25in
\begin{table}[h]
\begin{tabular}{ll}
2014 & \textbf{Lead Teaching Fellow}, \href{http://cs109.github.io/2014/}{Data Science (Harvard - CS109)} \\
& \textit{Held weekly lab sections. Developed course material, homework and solutions. } \\
& \textit{All material available on}:  \href{https://github.com/cs109/2014}{Github}\\
& \textit{Course taught by Rafael Irizarry and Verena Kaynig-Fittkau} \\
& \\
2014 & \textbf{Discussion Leader}, \href{https://www.edx.org/course/harvardx/harvardx-ph525x-data-analysis-genomics-1401}{Data Analysis for Genomics (HarvardX - PH525x)} \\
& \textit{Course taught by Rafael Irizarry and Mike Love} \\
& \\
2010, 2011 & \textbf{Teaching Assistant}, Probability in Bioinformatics and Genetics (Rice - STAT 423/623)   \\
& \textit{Grade homeworks, provide solutions, hold office hours and gave several guest lectures. } \\
& \\
2010 & \textbf{Teaching Assistant}, Applied Stochastic Processes (Rice - STAT 552)   \\
& \textit{Grade homeworks, provide solutions, hold office hours and gave several guest lectures. } \\
& \\
2008, 2009 & \textbf{Lab Instructor}, Introduction to Statistics for the Biosciences (Rice - STAT 305)   \\
 & \textit{Hold labs 3hr/week to teach students to use R.} \\
& \\
2007, 2008 & \textbf{Teaching Assistant}, Probability and Statistics (Rice - STAT 310)   \\
& \textit{Grade homeworks, provide solutions, hold office hours and gave several guest lectures. } \\
& \\
\end{tabular}
\end{table}

%Rice University \hfill 2007 - 2013 
%\begin{innerlist}
%        \item[] Teaching Assistant for Stat 423/623: Probability in Bioinformatics and Genetics 
%        \item[] Teaching Assistant for Stat 552: Applied Stochastic Processes
%        \item[] Teaching Assistant for Stat 310: Probability and Statistics
%        \item[] Lab Instructor for Stat 305: Introduction to Statistics for the Biosciences
%
%\end{innerlist}
%
%


        

%%%%%%%%%%%%%%%%%%%%%%%%%%%%%%%%%%%%%%
%%%%%%%%%%%%%% Software %%%%%%%%%%%%%%%%%%
%%%%%%%%%%%%%%%%%%%%%%%%%%%%%%%%%%%%%%


\section{Software}
\vskip -0.30in
\begin{table}[h]
\begin{tabular}{l}
\textit{quantro}: An R-package that can be used to test for differences between groups of distributions to \\
 \hspace{0.5in} guide the choice if quantile normalization should be used [\href{http://www.bioconductor.org/packages/devel/bioc/html/quantro.html}{Available on Bioconductor}]\\
\textit{epigenomeSim}: An R-package to simulate data from the epigenome including gene expression \\
\hspace{0.5in} and DNA methylation data \\
\textit{postMUT}: A tool implemented in Perl and R to predict the functionality of missense mutations \\
\textit{IBD2cond}: A tool implemented in Perl and R to predict regions of IBD in siblings \\
\end{tabular}
\end{table}





   

%%%%%%%%%%%%%%%%%%%%%%%%%%%%%%%%%%%%%
%%%%%%%%%%%%%% Service %%%%%%%%%%%%%%%%%%
%%%%%%%%%%%%%%%%%%%%%%%%%%%%%%%%%%%%%
%
%\section{Service}
%\vskip -.35in
%\begin{table}[h]
%\begin{tabular}{ll}
%2008 - 2012 & \textbf{Coordinator of the Computational Biology Seminar}, Rice University \\
%& \textit{Responsible for maintaining the website, scheduling and announcing the speakers.} \\
%& \\
%2008 - 2009 & \textbf{VP of Graduate Mathematics, Applied and Computational } \\
%& \hspace{.2in} \textbf{ Mathematicians and Statisticians}, Rice University \\
%& \textit{To foster an environment of growth and enthusiasm for math among the students }\\
%& \hspace{.2in} \textit{ through intellectual and social events.} \\
%& \\
%2005 - 2007 & \textbf{Treasurer of LSU Habitat for Humanity} \\
%\end{tabular}
%\end{table}
%
   

%%%%%%%%%%%%%%%%%%%%%%%%%%%%%%%%%%%%
%%%%%%%%%% Technical Skills %%%%%%%%%%%%%%%%%%
%%%%%%%%%%%%%%%%%%%%%%%%%%%%%%%%%%%%

\section{Technical Skills}
\textbf{Programming}: R, Python, Perl LaTeX{}, SAS, Matlab \\
\textbf{Operating Systems}: Mac OS X, Unix, Windows
   


\section{Statistical Expertise}
%
Statistical Learning, Statistical Genetics, Stochastic Processes,  Survival Analysis, Bayesian Data Analysis, Multivariate Analysis, Biostatistics, Probabilistic Models in Bioinformatics, General Linear Models, Statistical Sampling



        
%%%%%%%%%%%%%%%%%%%%%%%%%%%%%%%%%%%%
%%%%%%%%% Selected Memberships %%%%%%%%%%%%%%%
%%%%%%%%%%%%%%%%%%%%%%%%%%%%%%%%%%%%
   
\section{Professional Memberships}
%
Member, American Statistical Association (ASA) \\
Member, American Mathematical Society (AMS) \\
Member, American Society of Human Genetics (ASHG)




%
%\section{References}
%%
% Marek Kimmel, Ph.D. \\
%Professor, Department of Statistics \\ 
%Rice University \\
%kimmel@rice.edu, (713) 348-5255 \\
%
%Sharon Plon, M.D., Ph.D. \\
%Professor, Departments of Pediatrics and Human and Molecular Genetics \\
%Human Genome Sequencing Center \\
%Baylor College of Medicine \\
% splon@bcm.edu, (832) 824-4251 \\ 
% 
%Kathy Ensor, Ph.D. \\
%Professor and Chair, Department of Statistics \\
%Rice University \\
%ensor@rice.edu, (713) 348-4687  \\

\end{document}

%%%%%%%%%%%%%%%%%%%%%%%%%% End CV Document %%%%%%%%%%%%%%%%%%%%%%%%%%%%%